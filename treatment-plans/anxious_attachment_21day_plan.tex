% T21 Anxious Attachment Treatment Plan
% 21-Day Intensive Clinician Guide
% Branded for T21 Attachment Healing Program

\documentclass[11pt,letterpaper]{article}

% T21 Custom Style
\usepackage{t21_treatment_plan}
\usepackage{lastpage}

\begin{document}

% ============================================
% TITLE PAGE
% ============================================
\t21title{21-Day Anxious Attachment\\Treatment Plan}{Clinician Guide | Intensive Healing Protocol}

\thispagestyle{empty}

% ============================================
% EXECUTIVE SUMMARY - FIRST PAGE
% ============================================
\begin{patientinfo}
\begin{tabularx}{\textwidth}{lXlX}
\textbf{Program Type:} & Intensive Outpatient & \textbf{Duration:} & 21 Days \\
\textbf{Target Population:} & Adults with Anxious Attachment & \textbf{Format:} & Daily Sessions \\
\textbf{Treatment Setting:} & Individual + Group & \textbf{Evidence Base:} & AFT, EMDR, IFS, DBT \\
\end{tabularx}
\end{patientinfo}

\vspace{0.5cm}

\begin{goalbox}[Primary Treatment Goals (SMART)]
\begin{itemize}
    \item Reduce ECR-R attachment anxiety subscale score by 50\% within 21 days
    \item Develop and demonstrate 3 self-soothing techniques for abandonment triggers by Day 7
    \item Identify and cognitively restructure 5 core attachment beliefs by Day 14
    \item Establish and maintain daily secure self-attachment practices by Day 21
\end{itemize}
\end{goalbox}

\vspace{0.4cm}

\begin{keybox}[Core Therapeutic Modalities]
\begin{itemize}
    \item \textbf{Attachment-Focused Therapy (AFT):} Primary framework for understanding and healing attachment wounds
    \item \textbf{EMDR:} Processing early attachment trauma and relational injuries
    \item \textbf{Internal Family Systems (IFS):} Working with protective parts and exiled attachment needs
    \item \textbf{Somatic Experiencing:} Body-based regulation and nervous system healing
    \item \textbf{DBT Skills:} Distress tolerance and emotional regulation techniques
\end{itemize}
\end{keybox}

\vspace{0.4cm}

\begin{warningbox}[Critical Assessment Points]
\begin{itemize}
    \item Screen for trauma history, complex PTSD, and personality disorders at intake
    \item Assess suicide/self-harm risk---anxious attachment correlates with increased risk
    \item Monitor for relationship crises during treatment (common trigger)
    \item Evaluate co-occurring conditions: depression, anxiety disorders, substance use
\end{itemize}
\end{warningbox}

\newpage
\tableofcontents
\newpage

% ============================================
% SECTION 1: CLINICAL OVERVIEW
% ============================================
\section{Clinical Overview: Anxious Attachment}

\subsection{Diagnostic Presentation}

\textbf{Core Features of Anxious Attachment Pattern:}
\begin{itemize}
    \item Hyperactivation of attachment system in response to perceived threats
    \item Excessive proximity-seeking and reassurance-seeking behaviors
    \item Heightened vigilance to signs of rejection or abandonment
    \item Difficulty self-soothing without partner/attachment figure
    \item Tendency toward emotional dysregulation when attachment needs unmet
    \item Negative model of self, positive model of others (``I am unworthy, you can save me'')
\end{itemize}

\subsection{Common Presenting Complaints}

\begin{infobox}[Patient Presentation]
\begin{itemize}
    \item ``I can't stop worrying that my partner will leave me''
    \item ``I need constant reassurance that they still love me''
    \item ``When they don't respond to my texts, I panic''
    \item ``I know I'm too needy but I can't help it''
    \item ``I lose myself in relationships''
    \item ``I feel empty and worthless when I'm not in a relationship''
\end{itemize}
\end{infobox}

\subsection{Assessment Instruments}

\begin{t21table}{Assessment Battery}
\begin{tabularx}{\textwidth}{|l|X|l|}
\hline
\tableheadercolor
\textcolor{white}{\textbf{Instrument}} & \textcolor{white}{\textbf{Purpose}} & \textcolor{white}{\textbf{Timing}} \\
\hline
ECR-R (Experiences in Close Relationships-Revised) & Primary attachment style measure & Days 1, 11, 21 \\
\hline
\tablerowcolor
ASQ (Attachment Style Questionnaire) & Secondary attachment assessment & Days 1, 21 \\
\hline
PHQ-9 & Depression screening & Days 1, 11, 21 \\
\hline
\tablerowcolor
GAD-7 & Anxiety screening & Days 1, 11, 21 \\
\hline
PCL-5 & PTSD/trauma screening & Day 1 \\
\hline
\tablerowcolor
DERS (Difficulties in Emotion Regulation Scale) & Emotional regulation assessment & Days 1, 21 \\
\hline
RQ (Relationship Questionnaire) & Attachment prototype & Days 1, 21 \\
\hline
\end{tabularx}
\end{t21table}

% ============================================
% SECTION 2: TREATMENT FRAMEWORK
% ============================================
\section{Treatment Framework}

\subsection{Theoretical Integration}

This protocol integrates five evidence-based modalities:

\begin{phasebox}[Attachment-Focused Therapy (AFT)]
\textbf{Primary Framework}

AFT provides the overarching lens through which all interventions are delivered. Core principles:
\begin{itemize}
    \item Therapist as secure base for exploration
    \item Corrective emotional experiences within therapeutic relationship
    \item Making implicit attachment patterns explicit
    \item Developing earned secure attachment
\end{itemize}
\end{phasebox}

\vspace{0.4cm}

\begin{phasebox}[EMDR for Attachment Wounds]
\textbf{Trauma Processing}

Targeting early relational injuries using the Attachment-Focused EMDR protocol:
\begin{itemize}
    \item Process formative attachment memories (neglect, inconsistent caregiving)
    \item Desensitize triggers for abandonment fear
    \item Install positive cognitions about self-worth and lovability
    \item Resource development and installation before trauma processing
\end{itemize}
\end{phasebox}

\vspace{0.4cm}

\begin{phasebox}[Internal Family Systems (IFS)]
\textbf{Parts Work}

Working with internal parts that developed around attachment needs:
\begin{itemize}
    \item Identify protective ``manager'' parts (people-pleasing, hypervigilance)
    \item Access ``firefighter'' parts (protest behaviors, panic responses)
    \item Unburden ``exile'' parts carrying attachment wounds
    \item Strengthen Self-energy and self-leadership
\end{itemize}
\end{phasebox}

\vspace{0.4cm}

\begin{phasebox}[Somatic Experiencing]
\textbf{Body-Based Regulation}

Addressing nervous system dysregulation from attachment trauma:
\begin{itemize}
    \item Pendulation between activation and settling
    \item Titrated exposure to abandonment sensations
    \item Completing defensive responses
    \item Building somatic resources for self-regulation
\end{itemize}
\end{phasebox}

\vspace{0.4cm}

\begin{phasebox}[DBT Skills]
\textbf{Skill Building}

Practical skills for managing attachment distress:
\begin{itemize}
    \item Distress Tolerance: TIPP, STOP, radical acceptance
    \item Emotion Regulation: Opposite action, checking the facts
    \item Interpersonal Effectiveness: DEAR MAN, GIVE, FAST
    \item Mindfulness: Present-moment awareness, non-judgment
\end{itemize}
\end{phasebox}

% ============================================
% SECTION 3: 21-DAY PROTOCOL
% ============================================
\section{21-Day Treatment Protocol}

\weekdivider{WEEK 1: FOUNDATION \& STABILIZATION (Days 1--7)}

\subsection{Week 1 Overview}

\begin{goalbox}[Week 1 Goals]
\begin{itemize}
    \item Establish therapeutic alliance and secure base
    \item Complete comprehensive attachment assessment
    \item Psychoeducation on attachment theory and patterns
    \item Develop initial self-soothing toolkit (3 techniques)
    \item Begin nervous system regulation work
\end{itemize}
\end{goalbox}

\subsubsection{Day 1: Assessment \& Alliance}

\begin{daybox}[Day 1 --- Intake \& Assessment]
\textbf{Session Duration:} 90 minutes

\textbf{Objectives:}
\begin{itemize}
    \item Establish therapeutic frame and safety
    \item Complete attachment history intake
    \item Administer baseline assessments (ECR-R, PHQ-9, GAD-7, DERS, PCL-5)
    \item Discuss treatment goals and expectations
\end{itemize}

\textbf{Interventions:}
\begin{itemize}
    \item \intervention{AFT} Attachment history interview---explore primary caregiver relationships
    \item \intervention{IFS} Initial parts mapping---identify protective strategies
    \item \intervention{Somatic} Baseline nervous system assessment
\end{itemize}

\textbf{Homework:}
\begin{itemize}
    \item Attachment timeline exercise (key relational events)
    \item Daily mood/anxiety tracking log
    \item Identify 3 recent activating situations
\end{itemize}
\end{daybox}

\subsubsection{Day 2: Psychoeducation}

\begin{daybox}[Day 2 --- Understanding Anxious Attachment]
\textbf{Session Duration:} 60 minutes

\textbf{Objectives:}
\begin{itemize}
    \item Provide psychoeducation on attachment styles
    \item Help patient understand their attachment pattern
    \item Normalize anxious attachment as adaptive response
    \item Introduce concept of ``earned secure attachment''
\end{itemize}

\textbf{Key Teaching Points:}
\begin{itemize}
    \item Attachment system as survival mechanism
    \item Hyperactivation strategy: turning up the volume to get needs met
    \item How inconsistent caregiving creates anxious patterns
    \item Neurobiological basis of attachment anxiety
    \item Plasticity: attachment patterns can change
\end{itemize}

\textbf{Homework:}
\begin{itemize}
    \item Read handout on anxious attachment
    \item Identify 5 behaviors that stem from attachment anxiety
    \item Notice protest behaviors in daily life
\end{itemize}
\end{daybox}

\subsubsection{Day 3: Somatic Awareness}

\begin{daybox}[Day 3 --- Body-Based Awareness]
\textbf{Session Duration:} 60 minutes

\textbf{Objectives:}
\begin{itemize}
    \item Develop interoceptive awareness
    \item Identify somatic signatures of attachment anxiety
    \item Introduce grounding techniques
    \item Build body-based resources
\end{itemize}

\textbf{Interventions:}
\begin{itemize}
    \item \intervention{Somatic} Body scan---identify where attachment anxiety lives
    \item \intervention{Somatic} Pendulation practice---moving between activation and calm
    \item \intervention{DBT} Introduce TIPP skills (Temperature, Intense exercise, Paced breathing, Progressive relaxation)
\end{itemize}

\textbf{Homework:}
\begin{itemize}
    \item Daily body scan (5 minutes)
    \item Practice TIPP when activation arises
    \item Log physical sensations during triggering moments
\end{itemize}
\end{daybox}

\subsubsection{Day 4: Parts Introduction}

\begin{daybox}[Day 4 --- Meeting the Parts]
\textbf{Session Duration:} 60 minutes

\textbf{Objectives:}
\begin{itemize}
    \item Introduce IFS framework
    \item Begin parts mapping for attachment system
    \item Identify key protective parts
    \item Develop curiosity toward parts
\end{itemize}

\textbf{Interventions:}
\begin{itemize}
    \item \intervention{IFS} Guided parts mapping exercise
    \item \intervention{IFS} Identify the ``worrier,'' ``people-pleaser,'' ``clinger''
    \item \intervention{AFT} Connect parts to attachment experiences
\end{itemize}

\textbf{Common Parts in Anxious Attachment:}
\begin{itemize}
    \item \textbf{The Worrier:} Constantly scanning for signs of rejection
    \item \textbf{The People-Pleaser:} Sacrifices self to maintain connection
    \item \textbf{The Clingy Part:} Seeks constant reassurance and proximity
    \item \textbf{The Protester:} Creates drama to get attention
    \item \textbf{The Abandoned Child (Exile):} Holds the core wound
\end{itemize}

\textbf{Homework:}
\begin{itemize}
    \item Journal dialogue with one protective part
    \item Draw or visualize your parts system
    \item Notice which part gets activated in relationships
\end{itemize}
\end{daybox}

\subsubsection{Day 5: Self-Soothing Skills}

\begin{daybox}[Day 5 --- Building the Self-Soothing Toolkit]
\textbf{Session Duration:} 60 minutes

\textbf{Objectives:}
\begin{itemize}
    \item Develop 3 personalized self-soothing techniques
    \item Practice co-regulation in session
    \item Build capacity for self-regulation
    \item Create portable crisis toolkit
\end{itemize}

\textbf{Self-Soothing Techniques Menu:}
\begin{enumerate}
    \item \textbf{Bilateral Self-Hug:} Cross arms, tap alternately, breathe
    \item \textbf{Safe Place Visualization:} Detailed imaginal resource
    \item \textbf{Vagal Toning:} Cold water, humming, slow exhale
    \item \textbf{Grounding 5-4-3-2-1:} Sensory anchoring to present
    \item \textbf{Comfort Object:} Transitional object for soothing
    \item \textbf{Self-Compassion Phrases:} ``I am here for myself''
\end{enumerate}

\textbf{Homework:}
\begin{itemize}
    \item Practice each technique daily
    \item Create physical self-soothing kit
    \item Use one technique when activated---log effectiveness
\end{itemize}
\end{daybox}

\subsubsection{Day 6: Emotion Regulation}

\begin{daybox}[Day 6 --- DBT Skills for Attachment Distress]
\textbf{Session Duration:} 60 minutes

\textbf{Objectives:}
\begin{itemize}
    \item Introduce emotion regulation skills
    \item Practice ``Checking the Facts'' for attachment fears
    \item Learn ``Opposite Action'' for protest behaviors
    \item Develop distress tolerance for uncertainty
\end{itemize}

\textbf{Key Skills:}
\begin{itemize}
    \item \intervention{DBT} \textbf{Check the Facts:} ``Is my partner really leaving, or am I interpreting?''
    \item \intervention{DBT} \textbf{Opposite Action:} Instead of texting 10 times, practice sitting with the urge
    \item \intervention{DBT} \textbf{STOP Skill:} Stop, Take a step back, Observe, Proceed mindfully
    \item \intervention{DBT} \textbf{Radical Acceptance:} Accepting uncertainty in relationships
\end{itemize}

\textbf{Homework:}
\begin{itemize}
    \item Complete 3 ``Check the Facts'' worksheets on attachment fears
    \item Practice one Opposite Action per day
    \item Journal on radical acceptance of relationship uncertainty
\end{itemize}
\end{daybox}

\subsubsection{Day 7: Week 1 Integration}

\begin{daybox}[Day 7 --- Integration \& Reassessment]
\textbf{Session Duration:} 60 minutes

\textbf{Objectives:}
\begin{itemize}
    \item Review Week 1 learning and experiences
    \item Assess progress on self-soothing goal (3 techniques)
    \item Consolidate psychoeducation
    \item Prepare for trauma-focused work in Week 2
\end{itemize}

\textbf{Session Activities:}
\begin{itemize}
    \item Demo 3 self-soothing techniques \goaltag{Goal Check}
    \item Review homework logs and insights
    \item Address questions and resistance
    \item Preview EMDR and deeper parts work
\end{itemize}

\textbf{Homework (Weekend):}
\begin{itemize}
    \item Identify 5 core attachment beliefs for Week 2 work
    \item Continue daily self-soothing practice
    \item Write letter to younger self who first felt unlovable
\end{itemize}
\end{daybox}

\weekdivider{WEEK 2: PROCESSING \& RESTRUCTURING (Days 8--14)}

\subsection{Week 2 Overview}

\begin{goalbox}[Week 2 Goals]
\begin{itemize}
    \item Begin EMDR processing of attachment wounds
    \item Identify and challenge 5 core attachment beliefs
    \item Deepen IFS work---access and unburden exiles
    \item Develop new internal working models
    \item Build capacity to tolerate relationship uncertainty
\end{itemize}
\end{goalbox}

\subsubsection{Day 8: Core Beliefs Mapping}

\begin{daybox}[Day 8 --- Identifying Core Attachment Beliefs]
\textbf{Session Duration:} 60 minutes

\textbf{Objectives:}
\begin{itemize}
    \item Map negative core beliefs about self and others
    \item Connect beliefs to early attachment experiences
    \item Begin cognitive restructuring work
    \item Introduce positive cognitions for installation
\end{itemize}

\textbf{Common Anxious Attachment Beliefs:}
\begin{itemize}
    \item \textbf{About Self:} ``I am unlovable,'' ``I am too much,'' ``I am not enough,'' ``I am only worthy when needed''
    \item \textbf{About Others:} ``People will leave,'' ``I have to work hard to keep love,'' ``Their needs matter more''
    \item \textbf{About Relationships:} ``Love is scarce,'' ``I must be perfect to be loved,'' ``Conflict means abandonment''
\end{itemize}

\textbf{Desired Positive Cognitions:}
\begin{itemize}
    \item ``I am worthy of love as I am''
    \item ``I can trust myself to handle whatever happens''
    \item ``My needs matter''
    \item ``I am enough''
    \item ``I can survive separation''
\end{itemize}

\textbf{Homework:}
\begin{itemize}
    \item Identify 5 core beliefs to challenge
    \item Rate belief strength (1-7 VoC scale)
    \item Gather evidence for and against each belief
\end{itemize}
\end{daybox}

\subsubsection{Days 9-10: EMDR Processing}

\begin{daybox}[Days 9-10 --- EMDR for Attachment Trauma]
\textbf{Session Duration:} 90 minutes each

\textbf{Day 9 Objectives:}
\begin{itemize}
    \item Complete EMDR preparation (safe place, container, resourcing)
    \item Target first attachment memory
    \item Process through desensitization phase
\end{itemize}

\textbf{Day 10 Objectives:}
\begin{itemize}
    \item Continue processing or address new targets
    \item Install positive cognitions
    \item Body scan and closure
\end{itemize}

\textbf{Target Selection for Anxious Attachment:}
\begin{enumerate}
    \item Earliest memory of feeling abandoned/unseen
    \item Key rejection experience in childhood
    \item Formative experience of inconsistent caregiving
    \item First romantic rejection/abandonment
    \item Most distressing current trigger
\end{enumerate}

\textbf{EMDR Protocol Modifications:}
\begin{itemize}
    \item Extended resourcing before processing
    \item Use attachment figure interweaves (``What does the child need?'')
    \item Slow processing pace---monitor window of tolerance
    \item Frequent grounding breaks
\end{itemize}

\textbf{Post-Session Care:}
\begin{itemize}
    \item Provide grounding before leaving
    \item Review self-soothing techniques
    \item 24-hour check-in available
    \item Journal processing overnight
\end{itemize}
\end{daybox}

\subsubsection{Day 11: Parts Integration}

\begin{daybox}[Day 11 --- IFS Deep Work]
\textbf{Session Duration:} 60 minutes + Mid-Point Assessment

\textbf{Objectives:}
\begin{itemize}
    \item Access and begin unburdening exile parts
    \item Facilitate Self-to-exile connection
    \item Update protective parts on new capacity
    \item Administer mid-point assessments (ECR-R, PHQ-9, GAD-7)
\end{itemize}

\textbf{IFS Session Structure:}
\begin{enumerate}
    \item Check in with protective parts
    \item Request permission to access exile
    \item Witness the exile's story and pain
    \item Offer Self-energy, reparenting, and validation
    \item Begin unburdening ritual (release carried burdens)
    \item Update protectors on exile's healing
\end{enumerate}

\textbf{Homework:}
\begin{itemize}
    \item Journal dialogue with healed exile
    \item Notice changes in protector activity
    \item Practice Self-led responses to triggers
\end{itemize}
\end{daybox}

\subsubsection{Day 12: Cognitive Restructuring}

\begin{daybox}[Day 12 --- Challenging Core Beliefs]
\textbf{Session Duration:} 60 minutes

\textbf{Objectives:}
\begin{itemize}
    \item Systematically challenge identified core beliefs
    \item Develop alternative balanced beliefs
    \item Practice cognitive flexibility
    \item Build evidence base for new beliefs
\end{itemize}

\textbf{Cognitive Restructuring Process:}
\begin{enumerate}
    \item Identify the belief and rate strength
    \item Examine evidence for and against
    \item Identify cognitive distortions (mind-reading, catastrophizing)
    \item Develop balanced alternative
    \item Rate new belief strength
    \item Behavioral experiment to test new belief
\end{enumerate}

\textbf{Example Restructuring:}
\begin{itemize}
    \item \textbf{Old:} ``If I express my needs, they will leave''
    \item \textbf{Distortion:} Mind-reading, catastrophizing
    \item \textbf{Evidence against:} Times needs were expressed and met
    \item \textbf{New:} ``Expressing my needs helps build genuine connection. People who can't handle my needs aren't right for me.''
\end{itemize}

\textbf{Homework:}
\begin{itemize}
    \item Challenge remaining 2-3 beliefs using worksheet
    \item Collect evidence for new beliefs daily
    \item One behavioral experiment testing new belief
\end{itemize}
\end{daybox}

\subsubsection{Day 13: Relational Patterns}

\begin{daybox}[Day 13 --- Breaking Relational Cycles]
\textbf{Session Duration:} 60 minutes

\textbf{Objectives:}
\begin{itemize}
    \item Map typical relational cycle (trigger $\rightarrow$ behavior $\rightarrow$ consequence)
    \item Identify intervention points
    \item Practice new responses through role-play
    \item Develop exit ramps from anxious spirals
\end{itemize}

\textbf{Anxious Attachment Cycle:}
\begin{enumerate}
    \item \textbf{Trigger:} Partner seems distant
    \item \textbf{Interpretation:} ``They're pulling away, they don't love me''
    \item \textbf{Emotion:} Panic, anxiety, desperation
    \item \textbf{Behavior:} Seeking reassurance, checking, protest
    \item \textbf{Partner Response:} Overwhelmed, pulls back more
    \item \textbf{Confirmation:} ``See, I was right to worry''
    \item \textbf{Result:} Cycle intensifies
\end{enumerate}

\textbf{Intervention Points:}
\begin{itemize}
    \item \textbf{At Trigger:} Pause, ground, use STOP skill
    \item \textbf{At Interpretation:} Check the facts, challenge belief
    \item \textbf{At Emotion:} Self-soothe, tolerate distress
    \item \textbf{At Behavior:} Opposite action, delay response
\end{itemize}

\textbf{Homework:}
\begin{itemize}
    \item Map personal relational cycle
    \item Identify 3 exit ramps for each stage
    \item Practice one new response in relationship
\end{itemize}
\end{daybox}

\subsubsection{Day 14: Week 2 Integration}

\begin{daybox}[Day 14 --- Consolidation \& Goal Check]
\textbf{Session Duration:} 60 minutes

\textbf{Objectives:}
\begin{itemize}
    \item Review 5 core beliefs challenged \goaltag{Goal Check}
    \item Integrate EMDR processing
    \item Consolidate parts work
    \item Prepare for behavioral phase in Week 3
\end{itemize}

\textbf{Session Activities:}
\begin{itemize}
    \item Present 5 restructured beliefs with VoC ratings
    \item Process any residual disturbance from EMDR
    \item Check in with parts system---notice shifts
    \item Preview behavioral activation in Week 3
\end{itemize}

\textbf{Homework (Weekend):}
\begin{itemize}
    \item Complete belief restructuring summary
    \item Practice all 3 self-soothing techniques daily
    \item Write ``letter from secure self'' to current self
\end{itemize}
\end{daybox}

\weekdivider{WEEK 3: INTEGRATION \& SUSTAINABILITY (Days 15--21)}

\subsection{Week 3 Overview}

\begin{goalbox}[Week 3 Goals]
\begin{itemize}
    \item Establish daily secure self-attachment practices
    \item Apply skills to real-world relationship scenarios
    \item Build sustainable self-regulation capacity
    \item Develop relapse prevention plan
    \item Achieve 50\% reduction in ECR-R attachment anxiety score
\end{itemize}
\end{goalbox}

\subsubsection{Day 15: Secure Self-Attachment}

\begin{daybox}[Day 15 --- Becoming Your Own Secure Base]
\textbf{Session Duration:} 60 minutes

\textbf{Objectives:}
\begin{itemize}
    \item Introduce concept of secure self-attachment
    \item Develop personalized self-attachment rituals
    \item Practice internal secure base visualization
    \item Build capacity for self-validation
\end{itemize}

\textbf{Secure Self-Attachment Practices:}
\begin{enumerate}
    \item \textbf{Morning Grounding Ritual:} ``I am here for myself today''
    \item \textbf{Self-Validation Practice:} Acknowledge feelings without needing external confirmation
    \item \textbf{Inner Reparenting:} Dialogue with younger self, provide what was needed
    \item \textbf{Evening Integration:} Review day, celebrate showing up for self
    \item \textbf{Secure Base Visualization:} Internal safe figure who is always available
\end{enumerate}

\textbf{Homework:}
\begin{itemize}
    \item Establish morning and evening rituals
    \item Practice self-validation 3x daily
    \item Log moments of secure self-attachment
\end{itemize}
\end{daybox}

\subsubsection{Day 16: Interpersonal Skills}

\begin{daybox}[Day 16 --- Healthy Relating Skills]
\textbf{Session Duration:} 60 minutes

\textbf{Objectives:}
\begin{itemize}
    \item Teach assertive communication (DEAR MAN)
    \item Practice boundary-setting
    \item Develop capacity for healthy interdependence
    \item Differentiate needs from demands
\end{itemize}

\textbf{DBT Interpersonal Effectiveness:}
\begin{itemize}
    \item \intervention{DBT} \textbf{DEAR MAN:} Describe, Express, Assert, Reinforce, Mindful, Appear confident, Negotiate
    \item \intervention{DBT} \textbf{GIVE:} Gentle, Interested, Validate, Easy manner
    \item \intervention{DBT} \textbf{FAST:} Fair, Apologies (limited), Stick to values, Truthful
\end{itemize}

\textbf{Homework:}
\begin{itemize}
    \item Script one DEAR MAN request
    \item Practice one boundary-setting conversation
    \item Observe interdependence vs. codependence in relationships
\end{itemize}
\end{daybox}

\subsubsection{Day 17: Behavioral Experiments}

\begin{daybox}[Day 17 --- Testing New Beliefs in Action]
\textbf{Session Duration:} 60 minutes

\textbf{Objectives:}
\begin{itemize}
    \item Design behavioral experiments to test new beliefs
    \item Practice tolerating uncertainty without reassurance-seeking
    \item Build evidence for new internal working model
    \item Expand comfort zone gradually
\end{itemize}

\textbf{Sample Experiments:}
\begin{enumerate}
    \item Don't text partner first for 24 hours---observe outcome vs. predicted catastrophe
    \item Express a need directly and observe response
    \item Spend an evening alone without distraction---practice self-connection
    \item Let a conflict resolve naturally without pursuing resolution immediately
    \item Share vulnerability without immediately checking for reassurance
\end{enumerate}

\textbf{Experiment Protocol:}
\begin{enumerate}
    \item Identify belief to test
    \item Predict worst-case outcome (with anxious mind)
    \item Conduct experiment
    \item Record actual outcome
    \item Update beliefs based on evidence
\end{enumerate}

\textbf{Homework:}
\begin{itemize}
    \item Conduct 2-3 behavioral experiments
    \item Log predictions vs. outcomes
    \item Notice shifts in belief strength
\end{itemize}
\end{daybox}

\subsubsection{Day 18: Relapse Prevention}

\begin{daybox}[Day 18 --- Building Sustainability]
\textbf{Session Duration:} 60 minutes

\textbf{Objectives:}
\begin{itemize}
    \item Identify personal relapse warning signs
    \item Develop coping plan for high-risk situations
    \item Create maintenance plan for continued growth
    \item Establish ongoing support structure
\end{itemize}

\textbf{Warning Signs of Anxious Relapse:}
\begin{itemize}
    \item Increased reassurance-seeking frequency
    \item Returning to checking behaviors
    \item Difficulty being alone
    \item Neglecting self-care for partner's needs
    \item Catastrophic thinking about relationship security
\end{itemize}

\textbf{Coping Plan Components:}
\begin{enumerate}
    \item Recognize early warning sign
    \item Activate self-soothing protocol
    \item Check the facts on catastrophic thoughts
    \item Reach out to support (therapist, group, friend)
    \item Practice secure self-attachment ritual
    \item Recommit to boundaries and self-care
\end{enumerate}

\textbf{Homework:}
\begin{itemize}
    \item Complete written relapse prevention plan
    \item Identify 3 support people and their roles
    \item Schedule ongoing self-care practices
\end{itemize}
\end{daybox}

\subsubsection{Day 19: Integration Practice}

\begin{daybox}[Day 19 --- Putting It All Together]
\textbf{Session Duration:} 60 minutes

\textbf{Objectives:}
\begin{itemize}
    \item Practice integrated response to triggering scenario
    \item Role-play challenging situations
    \item Reinforce skill integration
    \item Build confidence in new patterns
\end{itemize}

\textbf{Integration Exercise:}

Present patient with triggering scenario and guide through integrated response:
\begin{enumerate}
    \item Notice trigger and somatic activation
    \item Ground using body-based technique
    \item Check in with activated parts
    \item Challenge automatic thoughts
    \item Choose secure self-attachment response
    \item Practice opposite action if needed
    \item Self-validate the experience
\end{enumerate}

\textbf{Homework:}
\begin{itemize}
    \item Apply integrated approach to real situation
    \item Journal the process step-by-step
    \item Note what worked and what needs practice
\end{itemize}
\end{daybox}

\subsubsection{Day 20: Future Self Work}

\begin{daybox}[Day 20 --- Embodying Secure Attachment]
\textbf{Session Duration:} 60 minutes

\textbf{Objectives:}
\begin{itemize}
    \item Connect with ``securely attached future self''
    \item Install resources for continued growth
    \item Strengthen positive internal working model
    \item Set intentions for ongoing journey
\end{itemize}

\textbf{Future Self Visualization:}

Guide patient in detailed visualization of securely attached self:
\begin{itemize}
    \item How does secure-self feel in their body?
    \item How do they respond to relationship uncertainty?
    \item What beliefs do they hold about love and worthiness?
    \item How do they show up in relationships?
    \item What advice do they have for current self?
\end{itemize}

\textbf{Resource Installation:}
\begin{itemize}
    \item Use EMDR bilateral stimulation to install future self as resource
    \item Anchor secure state somatically
    \item Create touchstone phrase or image for access
\end{itemize}

\textbf{Homework:}
\begin{itemize}
    \item Connect with future self daily
    \item Write letter from future self to current self
    \item Practice embodying secure posture/state
\end{itemize}
\end{daybox}

\subsubsection{Day 21: Completion \& Transition}

\begin{daybox}[Day 21 --- Program Completion]
\textbf{Session Duration:} 90 minutes

\textbf{Objectives:}
\begin{itemize}
    \item Complete final assessments (ECR-R, PHQ-9, GAD-7, DERS)
    \item Review treatment gains and goal achievement
    \item Celebrate transformation
    \item Establish aftercare plan
\end{itemize}

\textbf{Session Structure:}
\begin{enumerate}
    \item Administer final assessments
    \item Review progress on all SMART goals
    \item Process feelings about program completion
    \item Review relapse prevention plan
    \item Discuss continuing care options
    \item Closing ritual/celebration
\end{enumerate}

\textbf{Goal Achievement Review:}
\begin{itemize}
    \item[\goaltag{Goal 1}] ECR-R anxiety subscale reduction (target: 50\%)
    \item[\goaltag{Goal 2}] 3 self-soothing techniques demonstrated (Day 7)
    \item[\goaltag{Goal 3}] 5 core beliefs restructured (Day 14)
    \item[\goaltag{Goal 4}] Daily secure self-attachment practice established (Day 21)
\end{itemize}

\textbf{Aftercare Recommendations:}
\begin{itemize}
    \item Weekly/biweekly individual therapy (maintenance phase)
    \item Attachment-focused support group
    \item Monthly booster sessions for 3 months
    \item Daily secure self-attachment practice
    \item Continued skill practice and behavioral experiments
\end{itemize}
\end{daybox}

% ============================================
% SECTION 4: MONITORING & OUTCOMES
% ============================================
\section{Monitoring and Expected Outcomes}

\subsection{Progress Tracking}

\begin{t21table}{Outcome Monitoring Schedule}
\begin{tabularx}{\textwidth}{|l|c|c|c|X|}
\hline
\tableheadercolor
\textcolor{white}{\textbf{Measure}} & \textcolor{white}{\textbf{Day 1}} & \textcolor{white}{\textbf{Day 11}} & \textcolor{white}{\textbf{Day 21}} & \textcolor{white}{\textbf{Target}} \\
\hline
ECR-R Anxiety & Baseline & Check & Final & 50\% reduction \\
\hline
\tablerowcolor
PHQ-9 & Baseline & Check & Final & Score $<$10 \\
\hline
GAD-7 & Baseline & Check & Final & Score $<$8 \\
\hline
\tablerowcolor
DERS & Baseline & -- & Final & 30\% improvement \\
\hline
Self-Soothing Skills & 0/3 & -- & 3/3 & 3 techniques mastered \\
\hline
\tablerowcolor
Core Beliefs Restructured & 0/5 & -- & 5/5 & 5 beliefs challenged \\
\hline
Daily Practice & No & -- & Yes & Consistent practice \\
\hline
\end{tabularx}
\end{t21table}

\subsection{Expected Treatment Response}

\textbf{Week 1 (Days 1-7):}
\begin{itemize}
    \item Initial anxiety may increase (insight into patterns)
    \item Beginning of therapeutic alliance
    \item Some relief from psychoeducation (``there's a name for this'')
    \item First experiences of successful self-soothing
\end{itemize}

\textbf{Week 2 (Days 8-14):}
\begin{itemize}
    \item Temporary increase in distress during EMDR processing
    \item Gradual softening of core beliefs
    \item Moments of Self-energy access in IFS work
    \item Beginning cognitive flexibility
\end{itemize}

\textbf{Week 3 (Days 15-21):}
\begin{itemize}
    \item Integration and consolidation
    \item Behavioral evidence for new beliefs
    \item Increased capacity for self-regulation
    \item Emerging sense of secure self-attachment
\end{itemize}

% ============================================
% SECTION 5: CRISIS MANAGEMENT
% ============================================
\section{Crisis Management}

\begin{emergencybox}
\textbf{24/7 Crisis Resources:}
\begin{itemize}
    \item \textbf{988 Suicide \& Crisis Lifeline:} Call or text 988
    \item \textbf{Crisis Text Line:} Text HOME to 741741
    \item \textbf{International Association for Suicide Prevention:} https://www.iasp.info/resources/Crisis\_Centres/
\end{itemize}

\textbf{Clinician Emergency Protocol:}
\begin{itemize}
    \item Conduct suicide risk assessment if patient reports hopelessness or SI
    \item Anxious attachment patients at higher risk during relationship crises
    \item Safety plan required for all patients at intake
    \item After-hours coverage must be available during 21-day intensive
\end{itemize}
\end{emergencybox}

\subsection{High-Risk Periods}

\begin{warningbox}[Monitor Closely During:]
\begin{itemize}
    \item Relationship conflicts or breakups
    \item Partner unavailability (travel, busy periods)
    \item Major life transitions
    \item Days 9-10 during EMDR processing
    \item Program completion (fear of losing therapeutic attachment)
\end{itemize}
\end{warningbox}

% ============================================
% SECTION 6: PROVIDER NOTES
% ============================================
\section{Clinical Notes for Providers}

\subsection{Therapeutic Stance}

\begin{infobox}[Key Principles for Working with Anxious Attachment]
\begin{enumerate}
    \item \textbf{Be the secure base:} Consistent, reliable, attuned presence
    \item \textbf{Model rupture and repair:} Demonstrate that connection survives conflict
    \item \textbf{Avoid reinforcing reassurance-seeking:} Validate feelings, not catastrophic interpretations
    \item \textbf{Balance attunement with boundaries:} Show caring AND maintain therapeutic frame
    \item \textbf{Name the pattern, not the person:} ``Your anxiety is showing up'' vs. ``You're being anxious''
    \item \textbf{Expect transference:} Patient may replicate attachment pattern with therapist
    \item \textbf{Be direct about therapeutic relationship:} Use it as corrective experience
\end{enumerate}
\end{infobox}

\subsection{Common Challenges}

\begin{itemize}
    \item \textbf{Reassurance-seeking in session:} Redirect to internal resources
    \item \textbf{Cancellation anxiety:} Process as attachment trigger, maintain consistency
    \item \textbf{Idealization of therapist:} Gently reality-check while maintaining warmth
    \item \textbf{Resistance to ending:} Address early, plan gradual transition
    \item \textbf{Partner involvement requests:} Consider couples session, but maintain individual focus
\end{itemize}

\subsection{Contraindications}

\textbf{Screen Out or Modify Protocol For:}
\begin{itemize}
    \item Active suicidal ideation with plan/intent (stabilize first)
    \item Active substance use disorder (treat concurrently or first)
    \item Acute psychosis or severe dissociation
    \item Severe personality pathology requiring longer-term treatment
    \item Current abusive relationship (safety planning priority)
\end{itemize}

% ============================================
% SIGNATURE PAGE
% ============================================
\section{Attestation}

\vspace{1em}

\begin{tabular}{ll}
\textbf{Treating Clinician:} & \rule{8cm}{0.5pt} \\[1.5em]
\textbf{Credentials:} & \rule{8cm}{0.5pt} \\[1.5em]
\textbf{Date:} & \rule{5cm}{0.5pt} \\[2em]
\end{tabular}

\vspace{2em}

\begin{center}
\begin{tcolorbox}[
  enhanced,
  colback=t21cream50,
  colframe=t21primary,
  arc=4mm,
  boxrule=1pt,
  width=0.9\textwidth
]
\centering
\sffamily
\textbf{T21 Attachment Healing Program}\\[0.5em]
\small This treatment plan is designed for use by licensed mental health professionals.\\
Evidence-based interventions adapted for intensive 21-day format.\\[0.5em]
\textcolor{t21muted}{www.t21healing.com}
\end{tcolorbox}
\end{center}

\end{document}
